\nonstopmode{}
\documentclass[a4paper]{book}
\usepackage[times,inconsolata,hyper]{Rd}
\usepackage{makeidx}
\usepackage[utf8]{inputenc} % @SET ENCODING@
% \usepackage{graphicx} % @USE GRAPHICX@
\makeindex{}
\begin{document}
\chapter*{}
\begin{center}
{\textbf{\huge Scalable Cauchy Aggregate test using Multiple Phenotypes to test Interactions (SCAMPI)}}
\par\bigskip{\large \today}
\end{center}
\Rdcontents{\R{} topics documented:}
\inputencoding{utf8}
\HeaderA{CCT}{Cauchy Combination Test (CCT)}{CCT}
%
\begin{Description}
CCT is used to aggregate multiple p-value developed by Liu and Xie, 202, JASA, 115:529, 393-402, DOI: 10.1080/01621459.2018.1554485
The code used for implementing CCT is adopted from STAAR\_v0.9.7 contributed by Xihao Li and Zilin Li at
https://github.com/xihaoli/STAAR/blob/master/R/CCT.R
\end{Description}
%
\begin{Usage}
\begin{verbatim}
CCT(pvals, weights = NULL)
\end{verbatim}
\end{Usage}
%
\begin{Arguments}
\begin{ldescription}
\item[\code{pvals}] p-values, an array of p-values.

\item[\code{weights}] weights associated with each p-value, SCAMPI used the default weight, 1/len(pvals).
\end{ldescription}
\end{Arguments}
%
\begin{Value}
a single p-value aggregated from pvals.
\end{Value}
%
\begin{Examples}
\begin{ExampleCode}
set.seed(123)
pval <- runif(20)
CCT(pval)
\end{ExampleCode}
\end{Examples}
\inputencoding{utf8}
\HeaderA{DgLm}{Double Generalized Linear Models (DGLM) for Multiple Outcomes}{DgLm}
%
\begin{Description}
Apply Double Generalized Linear Models (DGLM) to standardize each
column in matrix y. The mean sub-model will be standardized by one genotype variable g\_var and multiple confounders z\_var;
The variance sub-model will be standardized by confounders z\_var. The confounders z\_var is optional.
\end{Description}
%
\begin{Usage}
\begin{verbatim}
DgLm(y, g_var, z_var = NULL)
\end{verbatim}
\end{Usage}
%
\begin{Arguments}
\begin{ldescription}
\item[\code{y}] outcome, a matrix with multiple columns, representing multiple traits

\item[\code{g\_var}] genotype, column matrix G with one column, representing one genotype variable

\item[\code{z\_var}] optional confounder, matrix Z with multiple columns, representing multiple confounders
\end{ldescription}
\end{Arguments}
%
\begin{Value}
standardized traits of the original trait matrix y, a matrix with multiple columns
\end{Value}
%
\begin{Examples}
\begin{ExampleCode}
DgLm(
  y = matrix(rnorm(500), ncol = 5),
  z_var = matrix(rnorm(400), ncol = 4),
  g_var = rbinom(100, 2, 0.25)
)
\end{ExampleCode}
\end{Examples}
\inputencoding{utf8}
\HeaderA{levene\_test}{Univariate Levene's Test}{levene.Rul.test}
%
\begin{Description}
The univariate Levene's test can be applied to a single trait. It is implemented
based on the formula in Pare and Cook, 2010, PLoS Genet 6(6): e1000981. doi:10.1371/journal.pgen.1000981
\end{Description}
%
\begin{Usage}
\begin{verbatim}
levene_test(y, g_var)
\end{verbatim}
\end{Usage}
%
\begin{Arguments}
\begin{ldescription}
\item[\code{y}] outcome, column matrix, representing one single trait

\item[\code{g\_var}] genotype, column matrix G with one column, representing one genotype variable
\end{ldescription}
\end{Arguments}
%
\begin{Value}
one single p-value
\end{Value}
%
\begin{Examples}
\begin{ExampleCode}
set.seed(123)
N <- 1000
G <- matrix(rbinom(N, 2, 0.25), ncol = 1)
W <- matrix(rnorm(N), ncol = 1)
Y <- 0.2 + 0.1 * G + 0.3 * W + 2 * G * W + rnorm(N)
levene_test(y = Y, g_var = G)
\end{ExampleCode}
\end{Examples}
\inputencoding{utf8}
\HeaderA{multivariate\_levene\_test}{Multivariate Levene's Test}{multivariate.Rul.levene.Rul.test}
%
\begin{Description}
Multivariate Levene's Test is extended in our paper from the univariate Levene's Test to
accomodate multiple traits.
\end{Description}
%
\begin{Usage}
\begin{verbatim}
multivariate_levene_test(y, g_var, z_var = NULL)
\end{verbatim}
\end{Usage}
%
\begin{Arguments}
\begin{ldescription}
\item[\code{y}] outcome, a matrix with multiple columns, representing multiple traits

\item[\code{g\_var}] genotype, column matrix G with one column, representing one genotype variable

\item[\code{z\_var}] optional confounder, matrix Z with multiple columns, representing multiple confounders
\end{ldescription}
\end{Arguments}
%
\begin{Value}
one aggregated p-value for Multivariate Levene's Test
\end{Value}
%
\begin{Examples}
\begin{ExampleCode}
set.seed(123)
multivariate_levene_test(
  y = matrix(rnorm(500), ncol = 5),
  z_var = matrix(rnorm(400), ncol = 4),
  g_var = matrix(rbinom(100, 2, 0.25), ncol = 1)
)
\end{ExampleCode}
\end{Examples}
\inputencoding{utf8}
\HeaderA{PC\_scampi\_test}{PC SCAMPI}{PC.Rul.scampi.Rul.test}
%
\begin{Description}
This is a variation of the SCAMPI method. PC SCAMPI treats the PC of traits as
the outcomes.
\end{Description}
%
\begin{Usage}
\begin{verbatim}
PC_scampi_test(y, g_var, z_var = NULL)
\end{verbatim}
\end{Usage}
%
\begin{Arguments}
\begin{ldescription}
\item[\code{y}] outcome, a matrix with multiple columns, representing multiple traits

\item[\code{g\_var}] genotype, column matrix G with one column, representing one genotype variable

\item[\code{z\_var}] optional confounder, matrix Z with multiple columns, representing multiple confounders
\end{ldescription}
\end{Arguments}
%
\begin{Value}
PC SCAMPI p-value
\end{Value}
%
\begin{Examples}
\begin{ExampleCode}
set.seed(123)
PC_scampi_test(
  y = matrix(rnorm(500), ncol = 5),
  z_var = matrix(rnorm(400), ncol = 4),
  g_var = matrix(rbinom(100, 2, 0.25), ncol = 1)
)
\end{ExampleCode}
\end{Examples}
\inputencoding{utf8}
\HeaderA{SCAMPI}{Scalable Cauchy Aggregate test using Multiple Phenotypes to test Interactions (SCAMPI)}{SCAMPI}
%
\begin{Description}
SCAMPI method is applied for detecting the G x E or G x G interaction effects by
utilizing both variance and covariance structure of multiple traits.
\end{Description}
%
\begin{Usage}
\begin{verbatim}
SCAMPI(y, g_var, z_var = NULL)
\end{verbatim}
\end{Usage}
%
\begin{Arguments}
\begin{ldescription}
\item[\code{y}] outcome, a matrix with multiple columns, representing multiple traits

\item[\code{g\_var}] genotype, column matrix G with one column, representing one genotype variable

\item[\code{z\_var}] optional confounder, matrix Z with multiple columns, representing multiple confounders
\end{ldescription}
\end{Arguments}
%
\begin{Value}
SCAMPI p-value
\end{Value}
%
\begin{Examples}
\begin{ExampleCode}
SCAMPI(
  y = matrix(rnorm(500), ncol = 5),
  z_var = matrix(rnorm(400), ncol = 4),
  g_var = rbinom(100, 2, 0.25)
)
\end{ExampleCode}
\end{Examples}
\printindex{}
\end{document}
